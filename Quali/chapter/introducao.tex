\chapter{Introdução}
Desde as mais antigas civilizações humanas, elas têm enfrentado o problema de propagação de doenças em larga escala \cite{historic}. Em 430 a.C aconteceu a epidemia chamada Peste de Atenas que foi responsável pela morte de cerca de 1/3 da população de Atenas. Em 541 houve a primeira pandemia chamada Praga de Justiniano que ocorreu no mediterrâneo, em 1347 aconteceu a mais devastadora pandemia na história da humanidade, a Peste Negra. Mais recentemente tivemos a Gripe Suína em 2009 e recentemente a Covid-19 e a Sétima Pandemia da Cólera. Nesse sentido o estudo de infecções se tornou cada vez mais necessário para cientistas seja para entender como uma infecção afeta o nosso corpo, seja para modelar a propagação dela.

Outrossim, com a expansão da humanidade nos últimos anos a partir do comércio, desmatamento e turismo facilitou a interação entre humanos e entre humanos e animais. Isso favoreceu uma maior propagação de doenças entre as civilizações \cite{area}. Essa propagação tem 4 classificações possíveis de acordo com a taxa de contágio e a sua área de atuação \cite{whats}.

\begin{itemize}
  \item \textbf{Endemia} significa que uma infecção tem taxa de contágio controlada e previsível que atua desde uma cidade até um continente;
  \item \textbf{Surto} expressa um aumento repentino na ocorrência de casos da doença em pequenas áreas;
  \item \textbf{Epidemia} é um surto em grande escala;
  \item \textbf{Pandemia} é uma epidemia em escala mundial.
\end{itemize}

\section{Contextualização da Problemática}