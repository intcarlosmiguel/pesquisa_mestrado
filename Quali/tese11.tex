%descomente abaixo para imprimir no formato de livro.
%\documentclass[ruledheader,11pt]{abnt-fisica9}
\documentclass[12pt]{abnt-fisica11}% opcao padrao, comente se desejar opcao livro
\usepackage[utf8]{inputenc}
\usepackage{algorithm}
\usepackage{algpseudocode}
\usepackage{import}
\usepackage{color, colortbl}
\usepackage{tikz}
\usepackage[brazil]{babel}
\usepackage{amsfonts}
\usepackage{multicol}
\usepackage{euscript}
\usepackage{float}
\usepackage{amsmath}
\usepackage{tikz,tkz-euclide}
\usetikzlibrary{arrows,calc,patterns}
\usepackage{epsfig,fisica-ufc11,graphicx,cite}
\usepackage{array}
\newcolumntype{P}[1]{>{\centering\arraybackslash}p{#1}}
	
\definecolor{Gray}{gray}{0.9}

\usepackage[num]{abntex2cite11}
\usepackage{mathrsfs}
% Use o bibtex
\newcommand{\mycomment}[1]{}
\usetikzlibrary{calc}
\tikzset{make origin horizontal center of bounding box/.style={%
execute at end picture={%
\path let \p1=(current bounding box.west),\p2=(current bounding box.east)
in ({-max(-1*\x1,\x2)},\y1) ({max(-1*\x1,\x2)},\y1);
}}}
% Opção alf para citações tipo (Author, data)


% descomente se quizer no padrão de livro papel b5:
%\estilobook

% se for  compiliar em pdflatex (aceita o brasao em pdf e gera o pdf ao inves do dvi)descomente a linha abaixo
\DeclareMathAlphabet{\mathpzc}{OT1}{pzc}{m}{it}
\pdflatextrue
\begin{document}

% essas informacoes do codigo CIP você consegue indo na biblioteca central
\codigocip{S185r}{CDD 530}%

\Ilustracao{1}% escolha 0 se não tiver ilustrações, 1 para ilustrações em petro e branco e 2 para ilustrações coloridas

\Tipo{1}%Escolha 0 para monografias, 1 para qualificação de mestrado, 2 para dissertação de mestrado,  3 para 
%qualificação de doutorado  e 4 para tese de doutorado

\autor{Carlos Miguel Moreira Gonçalves}%% Henri Poincaré

\autorr{Gonçalves, Carlos Miguel Moreira }%%Ex: Poincaré, Henri

\titulo{CESSANDO A PROPAGAÇÃO DE COVID-19 ELEMINANDO INDIVÍDUOS \textit{SUPER-SPREADERS} em Redes Complexas}
%\logodegrupo{ufc.jpg}% se houver logo de grupo
%\nomedogrupo{grupo de física teórica}% se necessário
%coloque ai as palavras chaves sobre a sua tese ou dissertaçao
%não esquecer o ponto depois das palavras chaves.
\pcs{Redes;}{Covid-19;}{Palavra-chave3;}{Palavra-chave4.}{}
 % the same but  in english
 \kws{Keyword1;}{Keyword2;}{Keyword3;}{Keyword4.}{}

\orientador{Prof. Dr. Leandro Chaves Rêgo}
%se for orientadora use \orientador[a]{Profa. Dra. nome da orientadora}%


\coorientador{Prof. Dr. Pablo Ignacio Fierens}
%se for co-orientadora use \coorientador[a]{Profa. Dra. nome da coorientadora}%




\dataaprovacao{07/03/2013}



\begin{bancaexaminadora}
  \assinatura{Prof. Dr. Leandro Chaves Rêgo (Orientador)\\ Universidade Federal do Ceará (UFC)}
    \assinatura{Prof. Dr. Pablo Ignacio Fierens (Coorientador)\\ Instituto Tecnológico de Buenos Aires (ITBA)}
    \assinatura{Prof. Dr. Sicrano de Tal \\ Universidade Federal do Piauí (UFPI)} 
  \end{bancaexaminadora}
  
  \Dedicatoria{“The properties of softness and hardness and darkness and clearness do not reside in the carbon atoms; they reside in the interconnections between the carbon atoms, or at least arise beacause of the interconnections between the carbon atoms.” - Nicholas Christakis}
  \begin{agradecimentos}
  A beltaro de tal bla bla bla bla bla bla bla bla blabla bla blabla bla blabla bla blabla bla blabla bla bla
  bla bla bla bla bla bla bla bla blabla bla blabla bla blabla bla blabla bla 
blabla bla bla  blabla bla blabla bla blabla.
  
  A fulano de tal  blabla bla blabla bla bla  blabla bla blabla bla bla  blabla 
bla blabla bla bla  blabla bla blabla bla bla blabla bla blabla bla blabla.
  \end{agradecimentos}
  \begin{resumo}
    A pandemia do Covid-19 foi um episódio adverso na história da humanidade causando um estrago na economia e na saúde mental de várias pessoas. Isso se deve muito ao perigo da doença antes da produção de vacinas e por causa de um isolamento forçado dentro de casa. Apesar de necessário, o isolamento social contribuiu para o agravamento desse cenário, colocar todas as pessoas em isolamento criou um ambiente de baixa produtividade, seja no Mercado de Trabalho e na Educação, por exemplo. Para evitar a quebra total, as instituições optaram pela utilização dos chamados \textit{home-office}, contudo ninguém estava preparado para uma mudança tão drástica na forma de viver. Será que o isolamento total de todos é necessário?

    Nesse trabalho, iremos estudar a remoção de indivíduos em redes levando em conta a sua importância na topologia e também considerando a taxa de mortalidade pelo Covid-19 de cada pessoa advinda da sua idade.
  \end{resumo}
  
  \begin{abstract}
  Write your abstract here bla bla bla bla bla bla bla bla blabla bla blabla bla 
blabla bla blabla bla blabla bla bla.
  \end{abstract}

\listadefiguras% comente se nao for necessario
 \listadegraficos
 % Na lista de figuras nao se deve por referencias.
 %se nao quizer comente as linhas antes do sumario
  \listadetabelas% comente se nao for necessario
  
  \listadesiglas% comente esta linha e as outras se nao for necessario
  \sigla{ABNT}{Associação Brasileira de Normas Técnicas}
  \sigla{CNPq}{Conselho Nacional de Desenvolvimento Científico e Tecnológico}
  \sigla{TESTE}{bla bla bla bla bla bla blabla bla blabla bla blabla bla blabla bla blabla bla bla bla bla bla bla bla bla blabla bla blabla bla blabla
  	blablabla bla blabla bla bla}
 \listadesimbolos
 
 
\simbolo{$\eta_{\mu\nu}$}{Métrica do espaço de Minkowsky}
\simbolo{$g_{\mu\nu}$}{Métrica do espaço curvo}
\simbolo{x}{bla bla}
 
 
 

 

 
\sumario

%\pagestyle{headings}

\import{./chapter/}{introducao}

\import{./chapter/}{fundamentos.tex}

\import{./chapter/}{metodologia.tex}

\chapter{conclus\~ao}
% \begin{figure}[h]
% 
% \center
% \subfigure{\includegraphics[width=0.45\textwidth]{Potentials135-D=5,10-FermionRight-caso2-eta1.pdf}} 
% \qquad
% \subfigure{\includegraphics[width=0.45\textwidth]{T135-D=5,10-FermionRight-caso2-eta1.pdf}}
% \caption{Potentials $U(z)$ and Transmission coefficient $T(m^2)$ for massive right fermions (odd dimensions) and Dirac fermions (even dimensions choosing $\xi_{n+}(x)$)
% with $\lambda=0,4$ for $D=5$ (line) and $D=10$ (dotted), with $\eta=1$, $M=1$ and $s=1$ (a, d), $s=3$ (b, e) and $s=5$ (c, f).}
% \label{Potentials3}
% \end{figure}

%\include{apenda}
%%%USAR BIBITEX
\bibliography{./bib/exemplo.bib}
% \begin{thebibliography}{5}
% \bibitem{weinberg:cosmology}
%   S.~Weinberg,
%    {\it Cosmology},
%   Oxford University Press Inc., New York (2008)
%  \end{thebibliography}

%%%%ATENÇÃO!!!!!
%%OS APÊNDICES E ANEXOS DEVER VIR DEPOIS DAS REFERÊNCIAS
%%USAR COMANDO \achapter ao invés de \chapter. SEÇÕES CONTINUAM COM COMANDOS NORMAIS
\apendice
\achapter{Testando o apêndice}
\section{Teste}
\subsection{teste}
\achapter{teste2}
\section{teste}
\anexo
\achapter{Testando o anexo}
\end{document}
