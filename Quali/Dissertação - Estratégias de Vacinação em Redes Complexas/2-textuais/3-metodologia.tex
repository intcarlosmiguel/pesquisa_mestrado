\chapter{Metodologia}

\section{Coleta de Dados}

A utilização de Redes Complexas como ferramenta de modelagem para estudos sobre a Covid-19 é intuitiva, pois o espalhamento das epidemias ocorre através de complexas redes de interações humanas. Para a implementação dessa abordagem, é essencial dispor de dados sobre redes de contágio que incluam informações relevantes para a análise, como idade, tempo de contato, pertencimento a grupos de risco, entre outros aspectos.

Entretanto, não é fácil encontrar um banco de dados tão completo e não enviesado pela recente pandemia. Em 2008, visando o estudo da pandemia de influenza, a Comissão Europeia criou o projeto chamado POLYMOD~\cite{POLYMOD} que tinha o intuito da coleta de dados para entender quais seriam os padrões de contatos da Europa~\cite{Mossong2008}. Apoiado nesse estudo, pesquisas posteriores se inspiraram no formato do POLYMOD~\cite{Belga2009,Belga2010,China,France,HongKong,Peru,Russia,Thailand,Vietnam,Zambia,Zimbabwe}. Essa pesquisa foi feita baseada em um questionário de maio de 2005 até setembro de 2006 em 8 diferentes países da Europa a partir de ligações aleatórias ou entrevistas presenciais. Os participantes da pesquisa foram recrutados de forma a serem amplamente representativos de toda a população em termos de distribuição geográfica, idade e sexo e houve uma super-amostragem para crianças e adolescentes.

 Ao final do dia a pessoa teria que preencher o diário anotando sobre informações pessoais, ambiente (escola, trabalho, casa, ...) e as pessoas na qual teve contato de curto alcance no dia, a idade aproximada (ou exata) dessas pessoas, duração da conversa, frequência na qual têm contato, dentre outras informações. Caso o participante fosse empregado, ele teria que informar qual a média de contatos no seu trabalho, caso fosse maior que 20 então os contatos que seriam anotados no diário seriam os não profissionais. 

Apesar dos dados obtidos a partir do estudo POLYMOD, é crucial notar que este não constrói uma rede de conexões entre os participantes. Tal limitação é inerente à metodologia adotada pelo estudo, que se baseia em discagem e entrevistas
aleatórias para coleta de informações. A escolha por esse método implica que as conexões entre os participantes não são mapeadas diretamente na forma de uma rede interconectada. A rede gerada pelos dados seria aproximadamente formada por vários grafos estrelas desconectados em que os nós centrais da estrela são os participantes das entrevistas. Portanto, para ser possível formar uma rede e utilizar os dados anteriores é necessário um modelo adequado que leve em consideração a distribuição de graus e a idade. Um modelo muito utilizado na literatura é o \textit{stochastic block model} (SBM) \cite{SBM} na qual cada nó seria separado em um bloco e os nós pertencentes a cada bloco seriam conectados com probabilidade $p_{B_\nu,B_\mu}$, na qual $B_\nu$ é o bloco de $\nu$. A principal desvantagem dessa abordagem é que a rede gerada apresenta uma característica média não resgatando, por exemplo, a distribuição de graus dos dados. \citeonline{klise2022prioritizing} apresenta uma outra forma de construir a rede na qual separa a rede em camadas relacionadas aos locais onde os indivíduos visitam (casa, escola ou trabalho) entretanto para a base de dados do POLYMOD isso não é possível pois não se tem noção da totalidade das redes.

\section{Modelo de Formação de Redes Proposto}

Para evitar esse problema~\cite{Manzo2020} propõe um modelo para formação de uma rede baseado 
no modelo de configuração para um problema similar. A partir dos dois dias de entrevista é calculado a média de contatos por dia de cada pessoa (arredondado para cima) e utiliza o Modelo de Configuração para a formação de uma Rede Sintética. No entanto, o modelo de formação de redes gera um agrupamento médio limitado e ele é muito importante na suavização de epidemias~\cite{Block2020}. Nesse sentido, 
Manzo e Rijt propõe que antes de ser excluído um dado nó do Modelo de Configuração (Algoritmo \ref{alg:manzo}), passamos por cada
par de vizinhos dele e os conectamos entre si com uma probabilidade $p$. Ou seja:

\begin{enumerate}
  \item Quando um dado nó 
  $\nu$
   atinge o grau requerido pelo MC são selecionados todos os vizinhos 
    $\eta(\nu)$;
    
  \item São escolhidos dois nós aleatoriamente
  $\mu,\zeta\in \eta(\nu)$ 
  e são conectados com probabilidade $p$;
  \item Se a ligação existir conectamos e salvamos para que ela não se repita no MC, caso contrário salvamos para que ela não se repita nesse algoritmo. Por fim também reduzimos em 1 o valor de $\kappa_\mu$ e $\kappa_\zeta$,
se a ligação entre $\mu$ e $\zeta$ for adicionada na rede;
  \item Repetimos 2 até que todas as ligações possíveis sejam testadas e o algoritmo termina para dar continuidade ao MC.
\end{enumerate}

\begin{algorithm}[htbp]
   \caption{Implementação de Manzo e van Rijt}
   \label{alg:manzo}
   \SetKwInOut{Input}{Input}
\Input{$G(\mathpzc{N} ,\mathpzc{L})$: rede já construída}
\Input{$\mathpzc{L}_\mathrm{test}$: possíveis ligações já testadas \tcp{Não são testadas duas vezes.}} 
\Input{$random()$: função que retorna um número ao acaso em $(0,1)$.}

   \For{$\nu\in \mathpzc{N}$}{
      \For{$\mu \in \eta(\nu)$}{
         \For{$\zeta \in \eta(\nu)\setminus\{\mu\}$ }{
            \If{$(\kappa_\zeta \neq 0)$ 
            $\mathbf{and }$ $(\kappa_\mu \neq 0)$}{
             \If{(random() <= $p)$}{
              \If{$(\mu,\zeta) \not\in \mathpzc{L}\cup \mathpzc{L}_\mathrm{test}$}{
                     $\mathpzc{L} \gets \mathpzc{L}\cup\{(\mu,\zeta)\}$ \tcp{Essa matriz vem do MC}
                     $\kappa_\mu \gets \kappa_\mu - 1$\\
                     $\kappa_\zeta \gets \kappa_\zeta - 1$\\
                  }
               }{
                  $\mathpzc{L}_\mathrm{test} \gets \mathpzc{L}_\mathrm{test}\cup\{(\mu,\zeta)\}$
               }
            }
         }
      }
   }
\end{algorithm}



\begin{figure}[H]
  \centering
  \captionsetup{font=normalsize,skip=0.8pt,singlelinecheck=on,labelsep=endash}
  \caption{Ilustração do Modelo de Manzo}
  \begin{tikzpicture}[make origin horizontal center of bounding box]
    %Values
    \def\xa{2};
    \def\ya{sqrt(4-sqrt(\xa))}

    \def\xc{0.5};
    \def\yc{sqrt(4-sqrt(\xc))};

    \def\xb{0.5};
    \def\yb{sqrt(4-sqrt(\xb))};

    \def\xd{1.5};
    \def\yd{sqrt(4-sqrt(\xd))};

    \def\xe{1.3};
    \def\ye{sqrt(4-sqrt(\xe))};

    \draw[black, thick] (0,0) -- ($\xa*(1,0) + \ya*(0,1)$);

    \draw[black, thick, dotted]($\xa*(1,0) + \ya*(0,1)$) -- ($\xa*(0.5,0) + \ya*(0,0.01)$);

    
    \draw[black, thick] (0,0) -- ($\xb*(1,0) + \yb*(0,1)$);
    \draw[black, thick, dotted] ($\xb*(1,0) + \yb*(0,1)$) -- ($\xb*(1,0) + \yb*(0,1.5)$);
    \draw[black, thick, dotted] ($\xb*(1,0) + \yb*(0,1)$) -- ($\xa*(1,0) + \ya*(0,1)$);
    \draw[black, thick, dotted] ($\xb*(1,0) + \yb*(0,1)$) -- ($\xc*(1,0.5) + \yc*(0.3,0.5)$);
    \draw[black, thick, dotted] ($\xb*(1,0) + \yb*(0,1)$) -- ($\xc*(1,0.5) + \yc*(-0.3,0.5)$);

    \draw[black, thick] (0,0) -- ($\xc*(1,0) + \yc*(0,-1)$);
    \draw[black, thick, dotted] ($\xc*(1,0) + \yc*(0,-1)$) -- ($\xa*(0.7,0) + \ya*(0,0.01)$);
    \draw[black, thick, dotted] ($\xc*(1,0) + \yc*(0,-1)$) -- ($\xc*(-1.5,0) + \yc*(0,-1)$);
    

    \draw[black, thick] (0,0) -- ($\xd*(-1,0) + \yd*(0,-1)$);
    \draw[black, thick, dotted] ($\xd*(-1,0) + \yd*(0,-1)$) -- ($\xd*(-1,0) + \yd*(0,0.01)$);
    \draw[black, thick, dotted] ($\xd*(-1,0) + \yd*(0,-1)$) -- ($\xd*(-1.8,0) + \yd*(0,-1)$);

    \draw[black, thick] (0,0) -- ($\xe*(0,1) + \ye*(-1,0)$);

    % Nós

    \draw[black, fill=white, anchor=center] (0,0) circle [radius=0.25] node {3};
    \draw[black, fill=white] ($\xc*(1,0) + \yc*(0,1)$) circle [radius=0.25] node {5};
    \draw[black, fill=white] ($\xe*(0,1) + \ye*(-1,0)$) circle [radius=0.25] node {0};
    \draw[black, fill=white] ($\xd*(-1,0) + \yd*(0,-1)$) circle [radius=0.25] node {80};
    \draw[black, fill=white] ($\xb*(1,0) + \yb*(0,-1)$) circle [radius=0.25] node {32};
    \draw[black, fill=white] ($\xa*(1,0) + \ya*(0,1)$) circle [radius=0.25] node {18};
    \node at (0,3) {$p = 0$};
  \end{tikzpicture}
  \begin{tikzpicture}[make origin horizontal center of bounding box]
    %Values
    \def\xa{2};
    \def\ya{sqrt(4-sqrt(\xa))}

    \def\xc{0.5};
    \def\yc{sqrt(4-sqrt(\xc))};

    \def\xb{0.5};
    \def\yb{sqrt(4-sqrt(\xb))};

    \def\xd{1.5};
    \def\yd{sqrt(4-sqrt(\xd))};

    \def\xe{1.3};
    \def\ye{sqrt(4-sqrt(\xe))};

    \draw[black, thick] (0,0) -- ($\xa*(1,0) + \ya*(0,1)$);

    \draw[black, thick, dotted]($\xa*(1,0) + \ya*(0,1)$) -- ($\xa*(0.5,0) + \ya*(0,0.01)$);

    
    \draw[black, thick] (0,0) -- ($\xb*(1,0) + \yb*(0,1)$);
    \draw[black, thick, dotted] ($\xb*(1,0) + \yb*(0,1)$) -- ($\xb*(1,0) + \yb*(0,1.5)$);
    \draw[black, thick] ($\xb*(1,0) + \yb*(0,1)$) -- ($\xa*(1,0) + \ya*(0,1)$);
    \draw[black, thick, dotted] ($\xb*(1,0) + \yb*(0,1)$) -- ($\xc*(1,0.5) + \yc*(0.3,0.5)$);
    \draw[black, thick, dotted] ($\xb*(1,0) + \yb*(0,1)$) -- ($\xc*(1,0.5) + \yc*(-0.3,0.5)$);

    \draw[black, thick] (0,0) -- ($\xc*(1,0) + \yc*(0,-1)$);
    \draw[black, thick, dotted] ($\xc*(1,0) + \yc*(0,-1)$) -- ($\xa*(0.7,0) + \ya*(0,0.01)$);
    %\draw[black, thick, dotted] ($\xc*(1,0) + \yc*(0,-1)$) -- ($\xc*(-1.5,0) + \yc*(0,-1)$);
    

    \draw[black, thick] (0,0) -- ($\xd*(-1,0) + \yd*(0,-1)$);
    \draw[black, thick] ($\xd*(-1,0) + \yd*(0,-1)$) -- ($\xc*(1,0) + \yc*(0,-1)$);
    \draw[black, thick, dotted] ($\xd*(-1,0) + \yd*(0,-1)$) -- ($\xd*(-1.8,0) + \yd*(0,-1)$);

    \draw[black, thick] (0,0) -- ($\xe*(0,1) + \ye*(-1,0)$);


    \draw[black, fill=white, anchor=center] (0,0) circle [radius=0.25] node {3};
    \draw[black, fill=white] ($\xc*(1,0) + \yc*(0,1)$) circle [radius=0.25] node {5};
    \draw[black, fill=white] ($\xe*(0,1) + \ye*(-1,0)$) circle [radius=0.25] node {0};
    \draw[black, fill=white] ($\xd*(-1,0) + \yd*(0,-1)$) circle [radius=0.25] node {80};
    \draw[black, fill=white] ($\xb*(1,0) + \yb*(0,-1)$) circle [radius=0.25] node {32};
    \draw[black, fill=white] ($\xa*(1,0) + \ya*(0,1)$) circle [radius=0.25] node {18};
    \node at (0,3) {$p = 0.5$};
  \end{tikzpicture}
  \begin{tikzpicture}[make origin horizontal center of bounding box]
    \def\xa{2};
    \def\ya{sqrt(4-sqrt(\xa))}

    \def\xc{0.5};
    \def\yc{sqrt(4-sqrt(\xc))};

    \def\xb{0.5};
    \def\yb{sqrt(4-sqrt(\xb))};

    \def\xd{1.5};
    \def\yd{sqrt(4-sqrt(\xd))};

    \def\xe{1.3};
    \def\ye{sqrt(4-sqrt(\xe))};

    \draw[black, thick] (0,0) -- ($\xa*(1,0) + \ya*(0,1)$);

    
    \draw[black, thick] (0,0) -- ($\xb*(1,0) + \yb*(0,1)$);
    \draw[black, thick, dotted] ($\xb*(1,0) + \yb*(0,1)$) -- ($\xb*(1,0) + \yb*(0,1.5)$);
    \draw[black, thick] ($\xb*(1,0) + \yb*(0,1)$) -- ($\xa*(1,0) + \ya*(0,1)$);
    \draw[black, thick, dotted] ($\xb*(1,0) + \yb*(0,1)$) -- ($\xc*(1,0.5) + \yc*(0.3,0.5)$);

    \draw[black, thick] (0,0) -- ($\xc*(1,0) + \yc*(0,-1)$);
    

    \draw[black, thick] (0,0) -- ($\xd*(-1,0) + \yd*(0,-1)$);
    \draw[black, thick] ($\xd*(-1,0) + \yd*(0,-1)$) -- ($\xc*(1,0) + \yc*(0,-1)$);
    \draw[black, thick] ($\xd*(-1,0) + \yd*(0,-1)$) -- ($\xb*(1,0) + \yb*(0,1)$);
    \draw[black, thick] ($\xb*(1,0) + \yb*(0,-1)$) -- ($\xa*(1,0) + \ya*(0,1)$);

    \draw[black, thick] (0,0) -- ($\xe*(0,1) + \ye*(-1,0)$);

    % Nós

    \draw[black, fill=white, anchor=center] (0,0) circle [radius=0.25] node {3};
    \draw[black, fill=white] ($\xc*(1,0) + \yc*(0,1)$) circle [radius=0.25] node {5};
    \draw[black, fill=white] ($\xe*(0,1) + \ye*(-1,0)$) circle [radius=0.25] node {0};
    \draw[black, fill=white] ($\xd*(-1,0) + \yd*(0,-1)$) circle [radius=0.25] node {80};
    \draw[black, fill=white] ($\xb*(1,0) + \yb*(0,-1)$) circle [radius=0.25] node {32};
    \draw[black, fill=white] ($\xa*(1,0) + \ya*(0,1)$) circle [radius=0.25] node {18};
    \node at (0,3) {$p = 1$};
  \end{tikzpicture}
  \captionsetup{font=small}
  \caption*{Funcionamento do Modelo de Configuração, 
  escolhemos dois nós $i$ e $j$ aleatoriamente e os conectamos. O Algoritmo pode gerar várias topologias de redes, porém ainda limitadas pelas quantidades $\{k_i\}$ de graus impostas pelo MC.\\ Fonte: Elaborado pelo autor}
  \label{img:MC_P}
\end{figure}

O modelo é mostrado na Figura \ref{img:MC_P}, quando um dado nó  $\nu$ atinge o grau necessário para cada $\mu \in \eta(\nu)$ 
os nós são conectados entre si com uma probabilidade $p$ se o grau de
$\nu$ não tiver atingido seu valor. Para que esse algoritmo se adapte a problemática utilizaremos os dados do POLYMOD. A partir deles é possível construir a distribuição $\{k_\nu\}_{\nu\in\mathpzc{N}}$ de graus e $\{f_\nu\}_{\nu\in\mathpzc{N}}$ de faixas etárias. 

Nesse contexto será proposto uma nova atualização no MC, consideraremos o conjunto 
$\kappa_{\nu,f}$ de conexões de um dado nó 
$\nu$ de faixa etária $f_\nu$ com um nó de faixa etária $f$, 
as quais serão incorporadas ao modelo de configuração, resultando em uma atualização considerando as conexões por faixa etária na qual será chamado de Modelo de Configuração Ponderado (MCP). Para a implementação do MCP, é fundamental dispor tanto da distribuição empírica de graus quanto de uma matriz $M$ tal que 
$M_{f,a}$ registra a proporção média de conexões de um nó na faixa etária $f$ que vão pra faixa etária $a$. 
Adicionalmente, a distribuição $F_w$ das faixas etárias é calculada a partir de um conjunto de dados. O procedimento para a implementação do MCP é delineado nos passos seguintes:
\begin{enumerate}
    \item São criados $N$ nós na rede formando-se o conjunto $\mathpzc{N}$;
    \item Atribui-se a cada nó $\nu$ uma faixa etária $f_\nu$, seguindo uma distribuição empírica $F_w$;
    \item Com base nos dados coletados, são geradas $k_\nu$ meias arestas para cada nó $\nu\in\mathpzc{N}$, seguindo a distribuição de graus empírica da faixa etária $f_\nu$;
    \item Para cada nó $\nu$, o número de meias arestas que vão se conectar com cada faixa etária $a$, $\kappa_{\nu,f}$, é escolhido conforme a distribuição multinomial \textbf{Multi}($k_\nu, M_{f_\nu,:}$);
    \item Os nós são ordenados em ordem decrescente de grau, de acordo com a sequência $\{\kappa_\nu\}$, para facilitar a convergência do modelo;
    \item Para cada nó $\nu$,uma faixa etária $a$ que ainda tenha meias arestas $\kappa_{\nu,a}$ a serem conectadas. Um nó $\mu$ dessa faixa etária é escolhido aleatoriamente entre todos os nós, excluindo $\nu$, que não possuam conexões com $\nu$, e que ainda tenham meias arestas disponíveis $\kappa_{\mu,f_\nu}$ para estabelecer; 
    \item Para cada nó $\mu$ selecionado dessa maneira, estabelece-se uma conexão com $\nu$, e os contadores $\kappa_{\mu,f_\nu}$ e $\kappa_{\nu,a}$ são decrementados em 1;
    \item O processo termina para o nó $\nu$ quando $\kappa_{\nu,a} = 0$ para todas as faixas $a$ ou não exista um nó $\nu_j$ que possa ser selecionado como descrito no passo 6;
    \item Logo após deve-se utilizar o Algoritmo de Manzo e van de Rijt adaptado com parâmetro $p$ para incrementar o agrupamento;
    \item Os passos de 6 a 9 são repetidos para todos os nós da rede.
\end{enumerate}

O algoritmo \ref{alg:MCP} é o pseudocódigo do MCP. Assim como o modelo tradicional o MCP tem um agrupamento limitado e baixo, portanto será implementado uma nova versão do Modelo de 
Manzo e van de Rijt para o MCP. Contudo ele precisa fazer algumas alterações para poder se adaptar ao MCP que utiliza as faixas etárias. O algoritmo \ref{alg:MCP_MANZO} é um pseudo-código com o algoritmo adaptado utilizando as faixas etárias, ele é chamado logo após todas as ligações 
$\{\kappa_{\nu,a}\}$ serem colocadas ou quando todos os sítios forem checados.


\begin{algorithm}[H]
   \caption{Implementação do MCP}
   \label{alg:MCP}
   $\mathpzc{L} \gets \emptyset$\\
   \For{$\nu\in\mathpzc{N}$}{
        $f_\nu \gets $ segundo a distribuição $F_w$\\
        $k_\nu \gets $ segundo a distribuição empírica de graus da faixa $f_\nu$\\
       $\{\kappa_{\nu,:}\}\gets $ segundo a distribuição multinomial \textbf{Multi}($k_\nu, M_{f_\nu,:}$)\\
       lista[$f_\nu$] $\gets \text{lista}[f_\nu] \cup \{\nu\}$\\
   }
   $\text{ordem} \gets $ordenar(\{$\nu$\},\{$k_\nu$\}) \tcp{Ordena os nós por número de meias arestas, em ordem decrescente.}
   \For{$\nu \in \mathrm{ordem}$}{
        \For{a < tamanho($F_w$)}{
            embaralha(lista[f])\\
            \For{$\mu\in\mathrm{lista}[a]\setminus \{\nu\}$}{
                \If{$\left[\kappa_{\mu,f_\nu} \neq 0\right]$ \textbf{and}  $\left[(\nu,\mu) \not\in \mathpzc{L}\right]$}{
                    $\mathpzc{L} \gets \mathpzc{L}\cup\{(\nu,\mu)\}$\\
                    $\kappa_{\nu,a} \gets \kappa_{\nu,a} - 1$\\
                    $\kappa_{\mu,f_\nu} \gets \kappa_{\mu,f_\nu} - 1$\\
                }
                \lIf{$\kappa_{\nu,a} == 0$}{
                    \textbf{break}
                }
            }
            Manzo\_Rijt($\nu,p$)
        }
   }
\end{algorithm}

\begin{algorithm}[htbp]
   \caption{Implementação de Manzo e van Rijt - Adaptado}
   \label{alg:MCP_MANZO}
   \SetKwInOut{Input}{Input}
\Input{$\nu$: nó sob consideração}
\Input{$p$: probabilidade de conexão} 
\Input{$G(\mathpzc{N} ,\mathpzc{L})$: rede já construída}
\Input{$\mathpzc{L}_\mathrm{test}$: possíveis ligações já testadas \tcp{Não são testadas duas vezes.}} 
\Input{$random()$: função que retorna um número ao acaso em $(0,1)$.}

   \For{$\mu \in \eta(\nu)$}{
        \If{$\text{soma}(\kappa_{\mu,:}) \neq 0$}{
            $\text{lista\_vizinhos}[f_\mu] \gets \text{lista\_vizinhos}[f_\mu] \cup \{\mu\}$\\
        }
   }

   \For{$\mu \in \eta(\nu)$}{
        \For{a < tamanho($F_w$)}{
            embaralha(lista\_vizinhos[$f$])\\
            \For{$\zeta \in \mathrm{lista\_vizinhos}[a]\setminus \{v\}$}{
                \lIf{$\kappa_{\mu,a} == 0$}{
                        \textbf{break}
                }
                \If{$\kappa_{\zeta,f_\mu} \neq 0$}{
                     \If{(random() <= $p)$}{
                      \If{$(\mu,\zeta) \not\in \mathpzc{L}\cup \mathpzc{L}_\mathrm{test}$}{
                             $\mathpzc{L} \gets \mathpzc{L}\cup\{(\mu,\zeta)\}$ \tcp{Essa matriz vem do MC}
                             $\kappa_{\mu,a} \gets \kappa_{\mu,a} - 1$\\
                             $\kappa_{\zeta,f_\mu} \gets \kappa_{\zeta,f_\mu} - 1$\\
                          }
                       }{
                          $\mathpzc{L}_\mathrm{test} \gets \mathpzc{L}_\mathrm{test}\cup\{(\mu,\zeta)\}$
                       }
                }
            }
        }
   }
\end{algorithm}

\section{Modelo de Infecção para COVID-19}

Artigos~\cite{Liu2022,Xiang2021} apresentam vários tipos de modelos na qual a literatura se utilizou para modelar a infecção e é importante entender qual é o melhor balanceamento de variáveis para não ficar muito complexo ou muito simplificado. Neste sentido, o modelo a ser utilizado é o 
SEIHARDS~\cite{Eikenberry2020} como base para nossa primeira aproximação. Ao contrário do paradigma inicial, a consideração da faixa etária dos indivíduos foi incorporada, reconhecendo-se as variações na suscetibilidade e resposta imunológicas associadas à idade. Além disso, a variável referente ao uso de máscaras faciais não foi incluída na nossa formulação, pois a ênfase está na análise de outros elementos de intervenção, o modelo epidemiológico adotado baseia-se em redes e houve a inclusão da possibilidade de recidiva à suscetibilidade após recuperação. Ele é composto pelos estágios: Suscetíveis (\textbf{S}), Expostos (\textbf{E}), Infectados Sintomáticos (\textbf{I}), Hospitalizados (\textbf{H}), Infectados Assintomáticos (\textbf{A}), Recuperados (\textbf{R}) e Mortos (\textbf{D}) que são modelados da seguinte forma:

\begin{itemize}
  \item \textbf{Suscetível (S)}: o nó neste estágio pode ser infectado pelos seus vizinhos que estão no estágio \textbf{A} ou no estágio \textbf{E}, essa taxa será $\epsilon_S$ vezes o número de contatos que estão no estágio \textbf{E} somado $\epsilon_A$ vezes o número de contatos no estágio A resultando na Equação \ref{eq:lambda} e um exemplo ilustrado é a Figura \ref{img:figinfect}.
  \begin{equation}
    \Lambda_\nu = \frac{|\mu \in \eta(\nu)\cap E|\cdot \epsilon_S + |\mu \in \eta(\nu)\cap A|\cdot \epsilon_A}{\langle k \rangle}
    \label{eq:lambda}
  \end{equation}

  \begin{figure}[ht]
  \centering
  
  
  \captionsetup{font=normalsize,skip=0.8pt,singlelinecheck=on,labelsep=endash}
  \caption{Ilustração do cálculo da taxa para se tornar infectado.}

  
  \begin{tikzpicture}[make origin horizontal center of bounding box]
    %Values
    \def\xa{2};
    \def\ya{sqrt(4-sqrt(\xa))}

    \def\xc{0.5};
    \def\yc{sqrt(4-sqrt(\xc))};

    \def\xb{0.5};
    \def\yb{sqrt(4-sqrt(\xb))};

    \def\xd{1.5};
    \def\yd{sqrt(4-sqrt(\xd))};

    \def\xe{1.3};
    \def\ye{sqrt(4-sqrt(\xe))};

    \draw[black, thick] (0,0) -- ($\xa*(1,0) + \ya*(0,1)$);
    
    \draw[black, thick] (0,0) -- ($\xb*(1,0) + \yb*(0,1)$);

    \draw[black, thick] (0,0) -- ($\xc*(1,0) + \yc*(0,-1)$);
    
    \draw[black, thick] (0,0) -- ($\xd*(-1,0) + \yd*(0,-1)$);

    \draw[black, thick] (0,0) -- ($\xe*(0,1) + \ye*(-1,0)$);

    % Nós

    \draw[black, fill=ninfect, anchor=center] (0,0) circle [radius=0.25] node {S};
    \draw[black, fill=infect] ($\xc*(1,0) + \yc*(0,1)$) circle [radius=0.25] node {E};
    \draw[black, fill=infect] ($\xe*(0,1) + \ye*(-1,0)$) circle [radius=0.25] node {E};
    \draw[black, fill=infect] ($\xd*(-1,0) + \yd*(0,-1)$) circle [radius=0.25] node {A};
    \draw[black, fill=ninteract] ($\xb*(1,0) + \yb*(0,-1)$) circle [radius=0.25] node {H};
    \draw[black, fill=ninteract] ($\xa*(1,0) + \ya*(0,1)$) circle [radius=0.25] node {H};
    \node at (0,3) {$\Lambda = 2\cdot\epsilon_S +1\cdot\epsilon_A = 2\cdot0.5+1\cdot0.41 = 1.41\, \mathrm{dia}^{-1}$};
  \end{tikzpicture}
  

  \label{img:figinfect}
\end{figure}

  \item \textbf{Exposto (E)}: um nó é considerado exposto quando ele estava no estágio \textbf{S} e foi contaminado por alguém no estágio \textbf{E} ou \textbf{A}. Este estágio é para representar os indivíduos que estão na fase de incubação do vírus, isso faz com que eles possam contaminar os outros indivíduos por não saberem que têm a doença. Deste estágio, ele pode passar para o estágio \textbf{I} com uma probabilidade 
  $\alpha^f$%PIF 16/5
, dependente da faixa etária $f$, 
  ou para o estágio \textbf{A} com probabilidade (1 - $\alpha^f$) mas com a mesma taxa $\sigma$.%;
  
  \item \textbf{Infectado Sintomático (I)}: um nó neste estágio está infectado, porém ele não pode infectar outras pessoas. Isso é feito para modelar um indivíduo que está sentindo os sintomas e, por isso, foi isolado e não teria mais contato com outras pessoas. 
  Uma pessoa neste estágio tem probabilidade $\psi^f$ de se hospitalizar, neste caso o tempo até hospitalização segue uma taxa $\phi^f$. Caso não seja necessária a hospitalização, a taxa do tempo de recuperação de sintomáticos é dada por $\gamma_I$.%;

  \item \textbf{Hospitalizado (H)}: nele o nó continua sem a capacidade de contaminar outros nós. Nesse estágio ele tem uma probabilidade $\tau^f$ de ir para o compartimento Morto com taxa $\delta$ e uma (1 - $\tau^f$) de se recuperar com taxa $\gamma_H$;
  
  \item \textbf{Infectado Assintomático (A)}: a diferença deste estágio para o \textbf{I} é o fato de que ele pode infectar outras pessoas. Isso se deve ao fato de que se ele é assintomático então não tem conhecimento da doença por não sentir nenhum sintoma, portanto ele continua tendo seus contatos e também não tem risco de morte ou necessidade de ser hospitalizado. Um indivíduo neste estágio se recupera com uma taxa $\gamma_A$.%;
  
  \item \textbf{Recuperado (R)}: caso o nó não morra ele será considerado recuperado na qual consegue voltar a ser suscetível novamente com uma taxa $\chi$.%;
  
  \item \textbf{Morto (D)}: Por fim, neste estado o nó perde todas as conexões e é nele que serão contabilizadas quantas pessoas morreram para utilizarmos como uma das métricas de minimização.
  
\end{itemize}

Na Figura \ref{img:Contagio} está esquematizado como funciona a dinâmica de infecção, sem considerar a vacinação das pessoas. 
No modelo, é assumido que o tempo de transição entre os estados segue uma distribuição exponencial com o parâmetro descrito como um rótulo das arestas na Figura~\ref{img:Contagio}. Os valores de cada parâmetro estão indicados nas Tabelas \ref{tabela:taxas_transicao_adaptada} e \ref{tabela:probabilidades_transicao_adaptada}.
Estes valores foram obtidos de 2 formas: ou pela busca em artigos ou utilizando a base de dados do openDataSUS para Síndrome Respiratória Aguda Grave de 2020~\cite{openDataSUSs}, um portal brasileiro na qual existem dados de internações Covid-19 e outras doenças gripais. Para achar $1/\gamma_H$, $1/\phi$, foi calculado a partir da média da diferença do tempo de uma pessoa ter sido internada para o tempo que ela saiu do hospital ou morreu, já a probabilidade $\tau^f$ foi obtida calculando quantas pessoas em uma faixa etária morreram dado que tinham sido internadas. Ademais, pelo fato de que o openDataSUS é limitado aos dados de internações, tornou-se necessária a busca de mais dados em artigos científicos.

\begin{figure}[ht]
  \centering

  \captionsetup{font=normalsize,skip=0.8pt,singlelinecheck=on,labelsep=endash}
  \caption{Esquematização do Modelo SEIHARDS de contágio}
  \begin{tikzpicture}[scale=0.9, every node/.style={scale=0.9}]

    \draw[->] (2,2.5) -- node[fill=white] {$\Lambda_\nu$} (3.9,2.5) ;
    \draw[->] (6,3.0) -- node[above,rotate=32]{$\sigma$} (7.9,4);
    \draw[->] (6,3.0) -- node[below,rotate=32]{\color{azuul}$1 - \alpha^f$} (7.9,4);
    

    \draw[->] (6,2.0) -- node[above,rotate=-32]{$\sigma$} (7.9,1);
    \draw[->] (6,2.0) -- node[below,rotate=-32]{\color{azuul}$\alpha^f$} (7.9,1);
    \draw[->] (10,4) -- node[fill=white] {$\gamma_A$} (12,5);

    \draw[->] (10,1) -- node[above] {$\phi$} (11.95,1);
    \draw[->] (10,1) -- node[below] {\color{azuul}$\psi^f$} (11.95,1);
    \draw[->] (14,1.) -- node[below] {\color{azuul} $\tau^f$} (16,1.);
    \draw[->] (14,1.) -- node[above] {$\delta$} (16,1.);
    \draw[->] (13,1.5) -- node[above,rotate=90] {\small $\gamma_H$} (13,4.5);
    \draw[->] (13,1.5) -- node[below,rotate=90] {\color{azuul}\small $1 - \tau^f$} (13,4.5);
    \draw[->] (10,1.5) -- node[above,rotate=45] {\small $\gamma_I$} (12,4.5);
    \draw[->] (10,1.5) -- node[below,rotate=45] {\color{azuul}\small $1 - \psi^f$} (12,4.5);
    
    \draw[->]  (13,5.) parabola (1,3.1);
    
    \node[draw,fill = white,draw = white] at (6,4.5) {$\chi$};
    
    
    \draw[rounded corners,drop shadow, fill=ninfect,draw=border, ultra thick] (0,2) rectangle (2,3)node[pos=.5,text= white,font=\fontsize{20}{20}\selectfont]  {$S_i$};
    
    \draw[rounded corners,drop shadow, fill=infect,draw=border, ultra thick](4, 2) rectangle (6,3) node[pos=.5,text= white,font=\fontsize{20}
    {20}\selectfont]  {$E_i$};
    
    \draw[rounded corners,drop shadow, fill=infect,draw=border, ultra thick](8, 3.5) rectangle (10,4.5) node[pos=.5,text= white,font=\fontsize{20}
    {20}\selectfont]  {$A_i$};
    
    \draw[rounded corners,drop shadow, fill=ninteract,draw=border, ultra thick](8, 0.5) rectangle (10,1.5) node[pos=.5,text= white,font=\fontsize{20}
    {20}\selectfont]  {$I_i$};
    
    \draw[rounded corners,drop shadow, fill=ninteract,draw=border, ultra thick](12, 0.5) rectangle (14,1.5) node[pos=.5,text= white,font=\fontsize{20}
    {20}\selectfont]  {$H_i$};
    
    \draw[rounded corners,drop shadow, fill=ninteract,draw=border, ultra thick](16, 0.5) rectangle (18,1.5) node[pos=.5,text= white,font=\fontsize{20}
    {20}\selectfont]  {$D_i$};
    
    \draw[rounded corners,drop shadow, fill=ninteract,draw=border, ultra thick](12, 5.5) rectangle (14,4.5) node[pos=.5,text= white,font=\fontsize{20}
    {20}\selectfont]  {$R_i$};
    
  \end{tikzpicture}
  %PIF27092023
  \captionsetup{font=small}
  \caption*{Ilustração do Modelo SEIHARD. Os blocos em vermelho significam quem pode contaminar outros, os blocos em azul escuro mostra o compartimento ainda não infectado e em azul claro mostra o estágio na qual o indivíduo não interage mais com a rede. Os parâmetros em preto são as taxas e em verde escuro são as probabilidades de entrar no estágio.}
  \label{img:Contagio}
\end{figure}

\begin{table}[H]
    \captionsetup{width=13.5cm}
    \caption{Taxas de transição utilizadas no modelo SEIAHRDS.}
    \centering
    \begin{tabular}{crrrrrrrr}
        \toprule
        Parâmetro & Definição & Valor [dia\(^{-1}\)] & Notas \\
        \midrule
        \midrule
        \(\epsilon_E\) & Taxa de contágio Expostos & 0.500 & \(^{(1)}\) \\
        \(\epsilon_A\) & Taxa de contágio Assintomática & 0.410 & \(^{(2)}\) \\
        \(\sigma\) & Taxa E\(\rightarrow\)A, E\(\rightarrow\)I (incubação) & 0.196 & \(^{(1)}\) \\
        \(\gamma_A\) & Taxa de recuperação de Assintomáticos & 0.143 & \(^{(1)}\) \\
        \(\gamma_I\) & Taxa de recuperação de Sintomáticos & 0.143 & \(^{(1)}\) \\
        \(\gamma_H\) & Taxa de recuperação de Hospitalizados & 0.083 & \(^{(3)}\) \\
        \(\phi\) & Taxa I\(\rightarrow\)H & 0.144 & \(^{(3)}\) \\
        \(\delta\) & Taxa de Mortalidade de Hospitalizados & 0.073 & \(^{(3)}\) \\
        \(\chi\) & Taxa R\(\rightarrow\)S (imunidade post-Covid) & 0.025 & \(^{(4)}\) \\
        \bottomrule
    \end{tabular}
    \caption*{Fontes: \(^{(1)}\)~\cite{Eikenberry2020}; \(^{(2)}\)~\cite{SP}; \(^{(3)}\) calculados usando o openDataSUS; \(^{(4)}\)~\cite{Kirkcaldy2020}.}
    \label{tabela:taxas_transicao_adaptada}
\end{table}


Considerando agora a vacinação, serão utilizados os dados da vacina da Pfizer-BioNTech, que apresenta uma grande eficácia \cite{HadjHassine2021} e 
foi a primeira vacina a ser utilizada. 
Como as vacinas só se tornaram disponíveis depois que a COVID-19 já tinha se espalhado por todas as regiões, adotaremos a convenção de que os indivíduos serão vacinados quando a difusão da doença já tenha entrado em um regime estacionário e o número de infectados tenha se estabilizado. Ao chegar nesse ponto são vacinados uma fração $f$ dos nós e devido à característica da vacina apenas os indivíduos nos estágios Suscetível, Recuperados ou Assintomáticos de acordo com um dado critério de prioridade. 
Com a vacinação, os parâmetros $\epsilon_S$, $\epsilon_A$, $\Lambda_\nu$, $\alpha^f$, $\psi^f$, e $\tau^f$ que influenciam a velocidade de propagação da infecção e a probabilidade de assintomatismo são alterados. No modelo, as mortes estão associadas aos indivíduos que passam pelo estágio sintomático, e esses parâmetros são críticos para a minimização do impacto. As modificações ocorrem por meio da multiplicação da eficácia da vacina, e os novos valores estão detalhados na Tabela \ref{tabela:vacina_adaptada}.


\begin{table}[H]
    \captionsetup{width=13.5cm}
    \caption{Probabilidades de transição por faixa etária.}
    \centering
    \begin{tabular}{crrrrrrr}
        \toprule
        Parâmetro & Definição & 0 - 20 & 20 - 30 & 30 - 50 & 50 - 70 & \(\geq\) 70 & Notas \\
        \midrule
        \midrule
        \(\alpha^f\) & Prob. E\(\rightarrow\)I & 29.10\% & 37.40\% & 41.68\% & 39.40\% & 31.30\% & \(^{(1)}\) \\
        \(\psi^f\) & Prob. I\(\rightarrow\)H & 0.408\% & 1.040\% & 3.890\% & 9.980\% & 17.500\% & \(^{(2)}\) \\
        \(\tau^f\) & Prob. H\(\rightarrow\)D & 1.040\% & 1.330\% & 1.380\% & 7.600\% & 24.000\% & \(^{(3)}\) \\
        \bottomrule
    \end{tabular}
    \caption*{Fontes: \(^{(1)}\)~\cite{Jung2020}; \(^{(2)}\)~\cite{RochaFilho2022}; \(^{(3)}\) calculada usando o openDataSUS.}
    \label{tabela:probabilidades_transicao_adaptada}
\end{table}


\begin{table}[H]
    \captionsetup{width=13.5cm}
    \caption{Mudança de parâmetros ao vacinar um nó.}
    \centering
    \begin{tabular}{crr}
        \toprule
        Definição & Não Vacinado & Vacinado  \\
        \midrule
        \midrule
        Taxa de contágio Sintomática & \(\epsilon_S\) & \(0.058 \times \epsilon_S\)\\
        Taxa de contágio Assintomática & \(\epsilon_A\) & \(0.058 \times \epsilon_A\)\\
        Prob. E\(\rightarrow\)I & \(\alpha^f\) & \(0.346 \times \alpha^f\)\\
        Prob. I\(\rightarrow\)H & \(\psi^f\) & \(0.034 \times \psi^f\) \\
        Prob. H\(\rightarrow\)D & \(\tau^f\) & \(0.034 \times \tau^f\)\\
        \bottomrule
    \end{tabular}
    \caption*{Valores das eficácias e sua interferência nos parâmetros. Fonte:~\cite{Haas2021}.}
    \label{tabela:vacina_adaptada}
\end{table}

