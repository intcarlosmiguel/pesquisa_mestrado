\chapter{Introdução}
Desde as mais antigas civilizações humanas, elas têm enfrentado o problema de propagação de doenças em larga escala~\cite{historic}. 
Por exemplo, em 
430 a.C. aconteceu a epidemia chamada Peste de Atenas que foi responsável pela morte de cerca de 
um terço  da população dessa cidade, em 
541 houve a primeira pandemia chamada Praga de Justiniano que ocorreu no mediterrâneo, 
e 
em 1347 aconteceu a mais devastadora pandemia na história da humanidade, a Peste Negra. Mais recentemente tivemos a Gripe Suína em 2009 e recentemente a Covid-19 e a Sétima Pandemia da Cólera. Nesse sentido, o estudo de 
doenças se tornou cada vez mais necessário para cientistas seja para entender como uma 
doença afeta o nosso corpo, seja para modelar a propagação dela.

Outrossim, com a expansão da humanidade nos últimos anos a partir do comércio, desmatamento e turismo facilitou a interação entre humanos e entre humanos e animais. Isso favoreceu uma maior propagação de doenças entre os países~\cite{area}. Essa propagação tem 4 classificações possíveis de acordo com a taxa de contágio e a sua área de atuação~\cite{whats}:

\begin{itemize}
  \item \textbf{Endemia} significa que uma infecção tem taxa de contágio controlada e previsível que atua desde uma cidade até um continente tem caráter contínuo e restrito a uma área geográfica ou população;
  \item \textbf{Surto} expressa um aumento repentino na ocorrência de casos da doença em pequenas áreas;
  \item \textbf{Epidemia} é um surto em grande escala;
  \item \textbf{Pandemia} é uma epidemia em escala mundial.
\end{itemize}

Dessa forma, a busca pelo entendimento das doenças emergiu como uma necessidade primordial desde os tempos mais remotos da história humana. Esse impulso de compreensão deu origem ao campo da Epidemiologia, uma disciplina voltada para a análise quantitativa e qualitativa dessas enfermidades. Essa área tem início com os estudos de Hipócrates~\cite{epidemiologia01} sobre como o ambiente 
favorece ou dificulta o surgimento de doenças. Apesar disso, foi apenas no século XIX que ganhou mais força com John Snow no estudo sobre a cólera em Londres. 
Snow descobriu que as mortes de cólera de 1848-49 e 1853-54 estavam relacionadas à água que os enfermos tomavam que era fornecida pela companhia Southpark. Outrossim, em 1760 Daniel Bernoulli foi o primeiro a tentar modelar 
matematicamente a propagação de doenças 
e as consequências da vacinação ~\cite{bernoulli2004attempt,Dietz2002}.
Já no século XX, os modelos compartimentais de transmissão de doenças foram introduzidos~\cite{kermack1927contribution} e 
 Richard Doll e Andrew Hill que descobriram a relação entre fumar e desenvolvimento de câncer~\cite{Doll1950}. 

Desde então a matemática mostrou-se cada vez mais essencial para estudo de doenças não apenas pela análise de dados, mas também na modelagem, pois não é eticamente correto fazer experimentações utilizando doenças, principalmente em humanos, então modelar matematicamente se torna essencial para previsões de propagação. Ademais, o entendimento da biologia por trás também é necessário para ter um modelo mais verossímil e seus resultados tenham 
validação na realidade.
Por fim, todo o conhecimento construído até aqui foi essencial para o entendimento e previsão da pandemia do novo Corona-vírus que durou de 11 de março de 2020 até 5 de maio de 2023.

\section{Covid-19}

Em dezembro de 2019, alguns casos de uma pneumonia de causas desconhecidas surgiram em Wuhan, capital da província de Hubei, um grande centro de transporte da China~\cite{Singhal2020}. 
Os pacientes que apresentaram esses sintomas faziam parte de um mercado de animais marinhos. Autoridades de saúde da China foram acionadas e foram coletadas amostras do patógeno para investigação 
e 
para caracterizar e controlar o desconhecido patógeno. No dia 31 de dezembro, o país notificou a OMS sobre o surto da doença e no dia seguinte o mercado da região foi fechado. No dia 7 de janeiro, cientistas conseguiram isolar o agente infecioso~\cite{Wang2020} e foi identificado com mais de 95\% de semelhança com o coronavírus apresentado em morcegos e mais de 70\% com SARS-CoV. No dia 23 de janeiro, já havia casos espalhados em 32 províncias da China e no mesmo dia toda a população de Wuhan foi colocada em \textit{lockdown}. Contudo, dois dias depois aconteceria a comemoração do Ano Novo Chinês e várias pessoas de outros países haviam viajado com intuito de participar do evento. No dia 11 de março, a OMS declara emergência de saúde pública do chamado Coronavírus ou COVID-19 e somente em
5 de maio de 2023 deixa de se tornar uma ameaça global.

Para evitar a proliferação do vírus, várias políticas públicas 
foram implementadas em todo o mundo. Estas políticas incluíam restrições de mobilidade, como \textit{lockdowns} e quarentenas, a fim de reduzir a interação social e a transmissão do vírus. Além disso, a promoção do uso de máscaras faciais, a intensificação da testagem e rastreamento de contatos, bem como a implementação de medidas de distanciamento físico em ambientes públicos, foram amplamente adotadas~\cite{PalaciosCruz2021}. 

No entanto, a centralização nas políticas de distanciamento social, que impuseram a necessidade de confinamento domiciliar à população, deu origem a desafios de natureza econômica~\cite{Irawan2021}. Isso se deveu, em grande parte, à impossibilidade de uma parcela substancial da força de trabalho desempenhar suas funções, resultando em um aumento significativo nas taxas de desemprego e na deterioração das condições sociais, com notável impacto negativo na qualidade da educação~\cite{10.1371/journal.pone.0239490}. O fechamento de instituições de ensino, como escolas e universidades, que são locais propensos a grandes aglomerações, agravou ainda mais essas questões. Como resultado desse contexto complexo, emergiu uma preocupação crítica com a saúde mental~\cite{Pereira2020}.


Devido a isso a campanha de vacinação em massa também se tornou uma política crucial para o fim da pandemia e minimizar 
o impacto dela 
na saúde pública e na economia. Um dos principais desafios da vacinação é a complexidade do processo de fabricação, que requer instalações especializadas, reagentes e tecnologia de ponta. Além disso, a demanda mundial por vacinas tem sobrecarregado a capacidade de produção existente, levando a atrasos na entrega de doses. A logística de distribuição também é complicada, com necessidade de armazenamento em temperaturas específicas para algumas vacinas, o que requer infraestrutura adequada. A obtenção de matérias-primas e ingredientes essenciais pode ser afetada por interrupções na cadeia de suprimentos global, causando a escassez de insumos.

\section{Estado da Arte}

Paralelamente aos avanços na prevenção da pandemia, houve um considerável aprimoramento dos modelos epidemiológicos, resultando em mais sofisticados e especializados na compreensão da propagação de doenças infecciosas~\cite{Xiang2021}. Esses modelos têm sido essenciais para orientar estratégias de controle da COVID-19, incluindo a implementação de programas de vacinação~\cite{Scabini2021, BustamanteCastaeda2021, Loyal2020}. No entanto, é importante observar que os modelos epidemiológicos tradicionais, embora sejam valiosos para prever tendências em larga escala, podem não capturar detalhes específicos das interações locais e complexas que ocorrem em estruturas de rede~\cite{Pellis2015}.

Os modelos epidemiológicos baseados em redes representam uma abordagem que enfatiza a análise das interações individuais em sistemas complexos, buscando compreender a propagação de doenças em escalas locais e com detalhes de rede. Esses modelos consideram a estrutura da rede, que descreve as conexões interpessoais e os contatos sociais, e incorporam processos de transmissão de doenças em estruturas de rede realistas~\cite{Pei2023}. \citeonline{pastor2002immunization,eames2003contact} utilizam o modelo de redes com propriedades de pequeno mundo para estudar a imunização e encontram expoentes críticos para a evolução da infecção. Enquanto que \citeonline{salathe2010dynamics,kitsak2010identification,Gong2013,miller2007effective} estudam formas de encontrar os sítios centrais na propagação das doenças utilizando algumas métricas de redes.


No entanto, é relevante destacar que essa abordagem ainda não incorporou completamente os avanços em complexidade ~\cite{Eikenberry2020} alcançados pelos modelos epidemiológicos 
que não utilizam informações de contatos ~\cite{Pellis2015}. Esses avanços estão relacionados à uma maior especialização em relação a doença com a introdução de novos componentes relevantes, como a distinção entre os sintomáticos e os assintomáticos, além de considerar aspectos relacionados à reação da sociedade, como a implementação de medidas de quarentena.

%PIF15042024 Neste parágrafo, os modelos são de agentes, né? Os modelos de agentes não são necessariamente modelos de redes.
Entretanto os modelos de redes têm avançado por outras frentes como por exemplo a vacinação de uma população de uma mesma doença com cepas diferentes~\cite{Li2023}, em estudo na qual a população se adapta à propagação da doença se utilizando de isolamento~\cite{Silva2023}, em estudo da importância de viagens entre cidades~\cite{Quiroga2023,DellaRossa2020}, do transporte~\cite{Scabini2021}, distanciamento social~\cite{Maheshwari2020} e agrupamento~\cite{Craig2020} para o espalhamento da doença. Além disso, ainda no que tange COVID e vacinação existem estudos utilizando redes e modelos de infecção para estudar o espalhamento da aceitação ou não aceitação da vacinação~\cite{n_vacina}.

Esses estudos sobre a COVID-19 têm explorado diversas abordagens para entender a dinâmica da pandemia, desde a previsão de curvas de infecção~\cite{Xiang2021} até a avaliação de estratégias de intervenção~\cite{Liu2022,kitsak2010identification}, como distanciamento social, \textit{lockdowns} e, principalmente, campanhas de vacinação. Através da análise de eficácia, cobertura vacinal e estratégias de implementação, é possível avaliar o impacto das vacinações em larga escala na redução da transmissão de patógenos e na proteção da saúde da população. Além disso, esses estudos contribuem para a tomada de decisões informadas sobre quais vacinas priorizar, quais grupos populacionais devem ser alvo e como otimizar recursos limitados. O estudo de estratégias de vacinação para COVID-19 não é novidade na literatura, ~\cite{Doostmohammadian2020,Tetteh2021,Chen2022,Petrizzelli2022}, entretanto os estudos estavam limitados a: redes tradicionais com pouca base no real, modelos epidemiológicos pouco especializados com a doença, sem levar consideração a idade do indivíduo ou tempo de contato entre eles e ou estudo de poucas métricas de centralidade.


\section{Descrição do Problema}

A incorporação de modelos epidemiológicos robustos em conjunto com a modelagem de redes, considerando a estratificação etária, representa um desenvolvimento metodológico que, em grande parte, não havia sido abordado satisfatoriamente na literatura. Essa abordagem combinada possibilita uma análise mais precisa e abrangente das interações sociais e da disseminação de doenças, levando em conta as diferenças demográficas relacionadas à idade. Ao fazer isso, abrem-se oportunidades para melhorar a modelagem de surtos epidêmicos e a eficácia das estratégias de intervenção, com foco na melhoria da vacinação em grupos etários específicos.


Vale ressaltar que, historicamente, a identificação de nós centrais em redes epidemiológicas dependia majoritariamente de métricas que têm a informação da estrutura da rede \cite{miller2007effective,kitsak2010identification,salathe2010dynamics}, alguns trabalhos recentes têm focado em abordagens que levam em consideração características dos indivíduos \cite{chen2021age,klise2022prioritizing}, mas ainda sim são bastante restritos.\\

Por fim, a implementação das campanhas de vacinação contra a COVID-19 em diferentes países revela, embora variem em detalhes, frequentemente refletem uma combinação de considerações etárias, econômicas e de exposição ao risco, em vez de uma priorização exclusiva focada na otimização da prevenção da disseminação do vírus e na redução da mortalidade \cite{Huh2021,Rosen2021,Cadeddu2022,Jung2021}. Enquanto alguns países, como Israel, adotaram critérios de vacinação simplificados, visando simultaneamente os indivíduos com maior risco de morte e de hospitalização, outros países europeus seguiram diretrizes da OMS para contextos de oferta limitada de vacinas, priorizando inicialmente profissionais de saúde e residentes de lares de idosos. 


\subsection{Objetivos Específicos}
\begin{itemize}
    \item Revisar a literatura sobre modelos epidemiológicos de COVID-19, sobre a rede de 
    contatos entre pessoas e suas faixas etárias e sobre os parâmetros que são necessários para modelar a propagação da doença e como a vacina interfere nisso;
    \item Gerar um modelo de redes que simule bem as nossas conexões sociais,
levando em conta os diferentes padrões de contatos para indivíduos de faixas etárias distintas;
    \item Encontrar um modelo apropriado para a propagação da doença, hospitalizações, doenças e efeitos da vacina;
    \item Fazer comparação de diversas estratégias de vacinação baseado em métricas de redes, propor novas métricas que combinem topologia e 
    propriedades dos nós.

\end{itemize}

\section{Organização do Trabalho}

O trabalho está estruturado em cinco capítulos. O segundo capítulo deste trabalho é dedicado a uma revisão teórica sobre redes, englobando a exploração de parâmetros essenciais, métricas de centralidade, diferentes modelos de redes e modelos epidemiológicos. Serão abordados conceitos fundamentais que fornecem a base teórica necessária para a compreensão da dinâmica das redes, sua estrutura e sua aplicabilidade em contextos epidemiológicos. Além disso, serão apresentados modelos que permitem simular a propagação de doenças em redes.

No terceiro capítulo, é detalhada a metodologia adotada para este estudo. Isso inclui uma descrição do banco de dados selecionado, justificando sua escolha e fornecendo informações relevantes sobre sua estrutura e conteúdo. É apresentado também o modelo de redes utilizado, destacando suas características e como ele foi aplicado para representar a interconexão entre indivíduos em um contexto específico. Além disso, o modelo epidemiológico escolhido é discutido em detalhes, assim como a integração dos dados provenientes da literatura sobre a COVID-19, os registros de vacinação da Pfizer-BioNTech e os dados do OpenDataSUS para informar e validar o modelo.

No quarto capítulo, são discutidos os resultados proeminentes da análise dos dados da pesquisa utilizada para construir o modelo em redes, resultado dos modelos de redes, epidemiológicos e das estratégias de vacinação apontando aquelas que foram melhores e como variaram perante um aumento do agrupamento e a ponderação ou não de arestas com o tempo de contato entre indivíduos. Nele mostramos os resultados de cada métrica de centralidade perante três aspectos: fração do número de mortos, tempo total hospitalizado e qual a fração de indivíduos é necessária para acabar com a doença. Um dos principais resultados é que o aumento do argumento tem uma mudança nos resultados da infecção enquanto que a ponderação nas arestas é algo que não altera significantemente os resultados, o \textit{Page Rank} foi a melhor centralidade apresentada e ela conseguiu acabar com a infecção com 45\% de pessoas vacinadas enquanto que a idade, o principal parâmetro usado como estratégia, foi preciso 85\%.


No último capítulo, será apresentados as conclusões da dissertação apresentando os resultados de forma resumida, as limitações desse projeto e quais são  os próximos passos para avançar nesta pesquisa.
