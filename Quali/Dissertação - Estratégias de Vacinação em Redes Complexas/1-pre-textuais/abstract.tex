The rapid spread, high mortality rate, and hospital overcrowding have made the SARS-CoV-19 virus a significant challenge for humanity. The emergent situation urgently called for investments in vaccine development as a critical measure to mitigate this issue. However, the financial costs and time required for vaccine development pose considerable obstacles to achieving comprehensive vaccination coverage. Therefore, understanding the most effective population immunization strategy is paramount to contain the virus's spread and optimize resource allocation in public policies. It is noteworthy that a significant portion of these policies was based on social distancing, which had significant impacts on the economic, educational, and mental health spheres.

The purpose of this study is to propose a modeling-based approach using the SEIRHADS contagion model on a complex contact network derived from a questionnaire-based survey. The objective is to identify the optimal vaccination strategy to reduce virus propagation, minimize hospitalizations, and mitigate the number of fatalities. The results obtained indicate that prioritizing vaccination for the elderly was more effective in preventing deaths and hospitalizations. On the other hand, adopting the strategy of vaccinating individuals with a higher number of contacts demonstrated better control in curbing contagion within the network.

% Separe as Keywords por ponto e vírgula.
\keywords{adult education; community schools; peasants; popular culture.}