A rápida disseminação, elevada taxa de mortalidade e sobrecarga nos sistemas hospitalares tornaram o vírus SARS-CoV-19 um desafio significativo para a humanidade. A emergência da situação exigia investimentos substanciais no desenvolvimento de vacinas como uma medida crucial para mitigar essa problemática. Entretanto, os custos financeiros e o tempo necessário para o desenvolvimento de vacinas representam obstáculos consideráveis para alcançar uma cobertura vacinal abrangente. Portanto, compreender a estratégia mais eficaz para a imunização da população é de suma importância para conter a propagação do vírus e otimizar a alocação de recursos em políticas públicas. É relevante destacar que grande parte dessas políticas estava fundamentada no distanciamento social, o qual teve impactos significativos nas esferas econômica, educacional e de saúde mental.

O propósito deste estudo consiste em propor uma abordagem baseada em modelagem, utilizando o modelo de contágio SEIHRADS em uma rede complexa de contatos derivada de uma pesquisa conduzida por meio de questionário. O objetivo é identificar a estratégia ótima de vacinação para reduzir a disseminação do vírus, minimizar as hospitalizações e mitigar o número de óbitos. Os resultados obtidos indicam que a estratégia de vacinar priorizando os mais velhos foi mais eficaz na prevenção de óbitos e hospitalizações. Por outro lado, ao adotar a estratégia de priorizar a vacinação dos indivíduos com maior número de contatos, observou-se um melhor controle na propagação do contágio dentro da rede.

% Separe as palavras-chave por ponto
\palavraschave{redes; covid-19; vacinação; medidas de centralidade.}