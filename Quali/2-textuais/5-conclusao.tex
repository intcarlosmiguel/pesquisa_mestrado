\chapter{Conclusões e Trabalhos Futuros}
\label{chap:conclusoes-e-trabalhos-futuros}

Neste trabalho, foram investigados diferentes métricas de centralidade e quais delas são melhores efetivas em identificar indivíduos chave na propagação do COVID-19. Ao estudar uma pesquisa conseguimos obter dados que nos ajudaram a construir um modelo de redes, apesar da pesquisa não contribuir para isso, propomos um modelo de formação de rede a partir de uma sequência de graus e as conexões entre indivíduos de faixa etária diferentes e propomos um modelo de propagação do COVID-19 e vacinação na rede.

%Resultados 
A partir do modelo utilizado, da variação do agrupamento, da ponderação da rede e das métricas a disposição foi mostrado que centralidades que consideraram peso nos nós e a estrutura da rede foram muito efetivas para vacinação contra COVID, essas estratégias conseguiam reduzir a mortalidade em de mais 60\% da mortalidade, 66\% do tempo total hospitalizado e foi necessário vacinar menos de 40\% da rede com a melhor estratégia para inibir a proliferação do vírus, ademais foi mostrado que utilizar peso nas arestas não teve ganho tão significativo, o que gera um menor gasto computacional, e até as vezes teve uma piora enquanto que uma abordagem altruísta dos valores se mostrou uma estratégia forte contra o vírus. Por fim, a classificação das melhores métricas mostrou ser bastante estável com o incremento do agrupamento da rede, o que possibilita não ser tão necessário a estimação esse valor em uma rede real.
    
%Próximos passos
Entretanto, essa pesquisa ainda apresenta limitações, o POLYMOD apesar de obter vários dados não se sabe a abrangência total e até que ponto se pode aplicar para diferentes países e culturas, o MCP não consegue aumentar o agrupamento para nós com grau maiores, não se sabe se é uma limitação do algoritmo, dos dados, ou é um comportamento padrão, o modelo epidemiológico que foi proposto abre margem para o estudo de superlotação de hospitais que foi uma característica importante na pandemia. Essas são uma das várias limitações do trabalho e propostas para futuras pesquisas nessa área.