\chapter{Conclusões e Trabalhos Futuros}
\label{chap:conclusoes-e-trabalhos-futuros}

Neste trabalho, foram investigados diferentes métricas de centralidade e quais delas são melhores efetivas em identificar indivíduos chave na propagação do COVID-19. Ao estudar uma pesquisa conseguimos obter dados que nos ajudaram a construir um modelo de redes, apesar da pesquisa não contribuir para isso, propomos um modelo de formação de rede a partir de uma sequência de graus e as conexões entre indivíduos de faixa etária diferentes e propomos um modelo de propagação do COVID-19 e vacinação na rede.

%Resultados 

%Próximos passos
Entretanto, essa pesquisa ainda apresenta limitações, o POLYMOD apesar de obter vários dados não se sabe a abrangência total e até que ponto se pode aplicar para diferentes países e culturas, o MCP não consegue aumentar o agrupamento para nós com grau maiores, não se sabe se é uma limitação do algoritmo, dos dados, ou é um comportamento padrão, o modelo epidemiológico que foi proposto abre margem para o estudo de superlotação de hospitais que foi uma característica importante na pandemia. Essas são uma das várias limitações do trabalho e propostas para futuras pesquisas nessa área.