The rapid spread, high mortality rate, and overload on hospital systems have made the SARS-CoV-19 virus a significant challenge for humanity. The emergency of the situation required substantial investments in vaccine development as a crucial measure to mitigate this issue. However, the financial costs and time needed for vaccine development represent considerable obstacles to achieving comprehensive vaccination coverage. Therefore, understanding the most effective strategy for population immunization is of paramount importance to contain the virus's spread and optimize resource allocation in public policies. It is relevant to highlight that many of these policies were based on social distancing, which had significant impacts on the economic, educational, and mental health spheres.

The purpose of this study is to propose a modeling-based approach using the SEIHRADS contagion model in a complex contact network derived from a survey conducted via questionnaire. The objective is to identify the optimal vaccination strategy to reduce the virus's spread, minimize hospitalizations, and mitigate the number of deaths. The study demonstrated the behavior and sensitivity of the infection model to variations in clustering and network construction, showing that the model is not very sensitive to clustering increments and minimally affected by edge weighting. Centrality measures that considered node weights and network structure proved highly effective for COVID-19 vaccination, with strategies like PageRank reducing mortality by over 60\%, total hospitalization time by 66\%, and requiring vaccination of less than 40\% of the network to inhibit virus proliferation. Additionally, using weighted edges did not yield significant computational gains and, in some cases, even worsened the results, while altruistic approaches emerged as strong strategies against the virus. The best metrics, the most effective metrics on average, and the Pareto frontier for the data were identified. Finally, the classification of the best metrics remained quite stable with the increase in network clustering, suggesting less need for precise estimates.

% Separe as Keywords por ponto e vírgula.
\keywords{adult education; community schools; peasants; popular culture.}