À minha noiva Isa Crisna Lima Duarte pelo apoio em todas as áreas da minha vida, seja no pessoal ou no acadêmico ela estava lá para me apoiar, tornou toda a luta diária um peso menor e me faz feliz. Aos meus pais e meu irmão que convivem comigo e me apoiam constantemente a melhorar e me tornar uma pessoa melhor.

O presente trabalho foi realizado com apoio da Coordenação de Aperfeiçoamento de Pessoal de Nível Superior – Brasil (CAPES) 

Ao Prof. Dr. Leandro Chaves Rêgo e ao Prof. Dr. Pablo Ignacio Fierens, pela excelente orientação, pelas conversas muito exclarecedoras e pelo apoio em toda a jornada acadêmica.

Aos professores participantes da banca examinadora Jesus Ossian da Cunha e Ricardo Lopes de Andrade pelo tempo, pelas valiosas colaborações e sugestões.

Aos professores Ascânio Dias Araújo, Michael Ferreira de Souza e por terem sedido o espaço do Laboratório de Simulação Numérica (LSN) para aprimorar meus estudos e simulações além das ótimas discussões sobre computação, algoritmos de Física.

Aos colegas do Laboratório de Simulação Numérica na qual aprendi muito com eles e fizeram com que todo o processo do mestrado e para o futuro fosse algo divertido acima de tudo.