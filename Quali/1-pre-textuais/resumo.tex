A rápida disseminação, elevada taxa de mortalidade e sobrecarga nos sistemas hospitalares tornaram o vírus SARS-CoV-19 um desafio significativo para a humanidade. A emergência da situação exigia investimentos substanciais no desenvolvimento de vacinas como uma medida crucial para mitigar essa problemática. Entretanto, os custos financeiros e o tempo necessário para o desenvolvimento de vacinas representam obstáculos consideráveis para alcançar uma cobertura vacinal abrangente. Portanto, compreender a estratégia mais eficaz para a imunização da população é de suma importância para conter a propagação do vírus e otimizar a alocação de recursos em políticas públicas. É relevante destacar que grande parte dessas políticas estava fundamentada no distanciamento social, o qual teve impactos significativos nas esferas econômica, educacional e de saúde mental.

O propósito deste estudo consiste em propor uma abordagem baseada em modelagem, utilizando o modelo de contágio SEIHRADS em uma rede complexa de contatos derivada de uma pesquisa conduzida por meio de questionário. O objetivo é identificar a estratégia ótima de vacinação para reduzir a disseminação do vírus, minimizar as hospitalizações e mitigar o número de óbitos. O estudo demonstrou o comportamento e a sensibilidade do modelo de infecção em relação às variações no agrupamento e na construção da rede, mostrando que o modelo é pouco sensível aos incrementos no agrupamento e pouco afetado pela ponderação das arestas. As medidas de centralidade que consideraram pesos nos nós e a estrutura da rede mostraram-se altamente eficazes para a vacinação contra a COVID-19, com estratégias como o PageRank reduzindo a mortalidade em mais de 60\%, o tempo total de hospitalização em 66\% e necessitando vacinar menos de 40 \% da rede para inibir a proliferação do vírus. Além disso, o uso de arestas ponderadas não resultou em ganhos computacionais significativos e, em alguns casos, até piorou os resultados, enquanto abordagens altruístas se mostraram estratégias fortes contra o vírus. Foram identificadas as melhores métricas, as métricas mais eficazes em média e a fronteira de Pareto para os dados. Por fim, a classificação das melhores métricas mostrou-se bastante estável com o incremento do agrupamento da rede, sugerindo uma menor necessidade de estimativas precisas.

% Separe as palavras-chave por ponto
\palavraschave{redes; covid-19; vacinação; medidas de centralidade.}